%%%%%%%%%%%%%%%%%%%%%%%%%%%%%%%%%%%%%%%%%%%%%%%%%%%%%%%%%%%%%%%%%%%%%%%%%%%%%%%%
% Global Pandoc Header file
% Author: Cooper Wallace
%
% Contains all my custom options to extend the default pandoc header
%%%%%%%%%%%%%%%%%%%%%%%%%%%%%%%%%%%%%%%%%%%%%%%%%%%%%%%%%%%%%%%%%%%%%%%%%%%%%%%%
% Additional Packages

\usepackage{graphicx}
\usepackage{xcolor}
\usepackage{geometry} 	% Change margin geometry
\geometry{margin=1in}
\usepackage{wrapfig}

\PassOptionsToPackage{fleqn}{amsmath}

\usepackage[version=3]{mhchem} 	% Chemical environment for Nuclear Notation
\usepackage{physics}	% Physics
\usepackage{svrsymbols}
\usepackage{mathtools}

%%%%%%%%%%%%%%%%%%%%%%%%%%%%%%%%%%%%%%%%%%%%%%%%%%%%%%%%%%%%%%%%%%%%%%%%%%%%%%%%
% Custom macros are defined in the 'customMacros.tex' file
% import needs to be used for file path outside cwd

% Math Macros
\usepackage{import}
\IfFileExists{/home/cooper/.pandoc/customMacros.tex}
	{\import{/home/cooper/.pandoc/}{customMacros.tex}}

% Reduce normal spacing
%\usepackage[nodisplayskipstretch]{setspace}
%\setstretch{1}

\setlength{\parskip}{0pt}
\setlength{\parindent}{0pt}

%%%%%%%%%%%%%%%%%%%%%%%%%%%%%%%%%%%%%%%%%%%%%%%%%%%%%%%%%%%%%%%%%%%%%%%%%%%%%%%%
% Custom Environments

% Allow Environment break over line
\allowdisplaybreaks

% Redefine proof environment to be indented automatically.
% https://tex.stackexchange.com/questions/279725/indent-proof-environment/279729#279729

\makeatletter
\renewenvironment{proof}[1][\proofname]{\par
  \pushQED{\qed}%
  \normalfont \topsep6\p@\@plus6\p@\relax
  \list{}{%
    \settowidth{\leftmargin}{\itshape\proofname:\hskip\labelsep}%
    \setlength{\labelwidth}{0pt}%
    \setlength{\itemindent}{-\leftmargin}%
  }%
  \item[\hskip\labelsep\itshape#1\@addpunct{:}]\ignorespaces
}{%
  \popQED\endlist\@endpefalse
}
\makeatother


\newenvironment{boxtext} {
	\vspace{0.5em}
	\begin{center}
    \begin{tabular}{|p{0.9\textwidth}|}
    \hline\\
    }
    {
    \\\\\hline
    \end{tabular}
    \end{center}
}
% Work around to allow Markdown formatting in LaTeX
\newcommand{\hideFromPandoc}[1]{#1}
\hideFromPandoc{
	\let\Begin\begin
	\let\End\end
}

%% Fancy Code Blocks
%% Source:
%%	https://tex.stackexchange.com/questions/179926/pandoc-markdown-to-pdf-without-cutting-off-code-block-lines-that-are-too-long

\lstset{
    basicstyle=\ttfamily,
    numbers=left,
    numberstyle=\footnotesize,
    numbersep=5pt,
	backgroundcolor=\color[RGB]{248,248,248},
	keywordstyle=\color[rgb]{0.13,0.29,0.53}\bfseries,
    stringstyle=\color[rgb]{0.31,0.60,0.02},
    commentstyle=\color[rgb]{0.56,0.35,0.01}\itshape,
    showspaces=false,
    showstringspaces=false,
    showtabs=false,
    tabsize=2,
    captionpos=b,
    breaklines=true,
    breakatwhitespace=true,
    breakautoindent=true,
    linewidth=\textwidth,
	frame=single
}
