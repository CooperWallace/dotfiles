%%%%%%%%%%%%%%%%%%%%%%%%%%%%%%%%%%%%%%%%%%%%%%%%%%%%%%%%%%%%%%%%%%%%%%%%%%%%%%%%
% Pandoc Header file
%
% Contains all my custom options to extend the default pandoc header
%%%%%%%%%%%%%%%%%%%%%%%%%%%%%%%%%%%%%%%%%%%%%%%%%%%%%%%%%%%%%%%%%%%%%%%%%%%%%%%%
% Additional Packages

\usepackage{graphicx}
\usepackage{xcolor}
\usepackage{geometry} 	% Change margin geometry
\geometry{margin=1in}

\usepackage{mhchem} 	% Chemical environment for Nuclear Notation


%%%%%%%%%%%%%%%%%%%%%%%%%%%%%%%%%%%%%%%%%%%%%%%%%%%%%%%%%%%%%%%%%%%%%%%%%%%%%%%%
% Custom macros are defined in the 'customMacros.tex' file
% import needs to be used for file path outside cwd

% Math Macros
\usepackage{import}
\IfFileExists{/home/cooper/.pandoc/customMacros.tex}
	{\import{/home/cooper/.pandoc/}{customMacros.tex}}


%%%%%%%%%%%%%%%%%%%%%%%%%%%%%%%%%%%%%%%%%%%%%%%%%%%%%%%%%%%%%%%%%%%%%%%%%%%%%%%%
% Custom Environments

\newenvironment{boxtext} {
	\vspace{0.5em}
	\begin{center}
    \begin{tabular}{|p{0.9\textwidth}|}
    \hline\\
    }
    {
    \\\\\hline
    \end{tabular}
    \end{center}
}
% Work around to allow Markdown formatting in LaTeX
\newcommand{\hideFromPandoc}[1]{#1}
\hideFromPandoc{
	\let\Begin\begin
	\let\End\end
}


% This is a macro to include lecture slides inside of my notes
% to avoid repeating them

% Nice command to define lecture slide location
\newcommand{\noteSlides}[1] {
	\newcommand \LectSlides {#1}
}

\newenvironment{lectSlide}[2] {
	\begin{center}
	\fbox{\includegraphics[page=#1, width=0.45\linewidth, clip]{\LectSlides}
	\includegraphics[page=#2, width=0.45\linewidth, clip]{\LectSlides}}
	\end{center}
}


\newenvironment{lectSlideiii}[3] {
	\begin{center}
	\fbox{
		\includegraphics[page=#1, width=0.32\linewidth, clip]{\LectSlides}
		\includegraphics[page=#2, width=0.32\linewidth, clip]{\LectSlides}
		\includegraphics[page=#3, width=0.32\linewidth, clip]{\LectSlides}
	}
	\end{center}
}


%% Fancy Code Blocks
%% Source:
%%	https://tex.stackexchange.com/questions/179926/pandoc-markdown-to-pdf-without-cutting-off-code-block-lines-that-are-too-long

\lstset{
    basicstyle=\ttfamily,
    numbers=left,
    numberstyle=\footnotesize,
    numbersep=5pt,
	backgroundcolor=\color[RGB]{248,248,248},
	keywordstyle=\color[rgb]{0.13,0.29,0.53}\bfseries,
    stringstyle=\color[rgb]{0.31,0.60,0.02},
    commentstyle=\color[rgb]{0.56,0.35,0.01}\itshape,
    showspaces=false,
    showstringspaces=false,
    showtabs=false,
    tabsize=2,
    captionpos=b,
    breaklines=true,
    breakatwhitespace=true,
    breakautoindent=true,
    linewidth=\textwidth,
	frame=single
}
